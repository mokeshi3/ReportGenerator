本章では、プレ卒の研究を通じて得られた知見から、今後の展望について述べる。

\subsection{先行研究のアルゴリズムの改善}
\ref{s:改善案}章で述べたように、先行研究のアルゴリズムには改善の余地がある。これらの改善により、点字の認識精度の向上することが期待される。精度の向上以外の改善点として、現在のアルゴリズムの利便性を向上させることが考えられる。現在のアルゴリズムは圧力センサからの入力は定規を利用して、水平に入力されることが前提となっている。これは利便性の面で問題があるため、これを改善し、斜め入力、多少のブレの許容などに対応することが望ましい。しかし、これらに対応するには、現在のアルゴリズムの大部分を変更が必要であることが予想される。

\subsection{新たな手法の研究}
先行研究のアルゴリズムを改善することで精度を向上させることは可能であると考えられるが、斜め入力などへの対応は非常に困難である。そこで、得られた知見を利用し、新たな手法による点字認識を検討している。

その1つが特徴マップである。特徴マップとは、物体検出の分野で用いられている手法である。特徴マップは図\ref{f:特徴マップ}のようにデータの領域をセルと呼ばれる単位で分割し、セルに対象の物体を含まれているかを判定する手法である。これを利用することで、圧力データの入力に対して、ある程度の柔軟性を持たせられると考えられる。
\begin{figure}[H]
	\centering
	\includegraphics[width=7.0cm]{img/特徴マップ.png}
	\caption{特徴マップのイメージ}
	\label{f:特徴マップ}
\end{figure}
図\ref{f:特徴マップ}は4$\times$4に分割され、点線の長方形で1つの領域が囲まれていることがわかる。点線は、各領域に凸点が存在するかを判定するフィルタを表している。このように分割した各領域に凸点があるかを複数のフィルタを利用して検証することで、ある程度の柔軟性を持たせられると考えられる。

または、入力データを圧縮することが考えられる。入力データを圧縮することで、多少のブレなどは圧縮する際に解消することができると考えられる。さらには、ノイズなども圧縮により解消することができると考えられる。

その他の手法として、AIの利用が考えられる。先行研究のアルゴリズムの精度の向上に成功した場合、先行研究のアルゴリズムによって得られて点字を教師データとしてAIに与えることで、圧力データの入力に柔軟性を持たせられると考えられる。
