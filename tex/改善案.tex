本章では\ref{s:誤認識の原因}章で述べた誤認識の原因の解決案について述べる。

\subsection{圧力データの正規化}
点字認識の精度を上げる手段の1つとして圧力データの正規化が考えられる。圧力センサを利用して点字を読み込むとき、使用者により圧力の値や点字の入力速度が変化することが考えられる。この場合、現在のアルゴリズムのように圧力データの値をそのまま利用することおよびパラメータを固定値として利用することにより誤認識が発生しやすくなると考えられる。このことから、圧力データを正規化することとパラメータを動的に決定することが考えられる。

圧力データの正規化として考えられるのは、入力データの値を0から1の小数に正規化することがである。現在のアルゴリズムでは圧力データの値をそのまま利用しているため、530から65535の値を取ることがわかっている。入力の大部分は10000に満たない値であるが、ノイズにより一部の値のみが特出して高くなることが考えられる。この対策として、標準化のように分散を考慮した方法が望ましいと考えられる。

\subsection{パラメータの動的な決定}
現在のアルゴリズムでは、行座標の決定および列座標の決定する際に、特定の値になっているかどうかなどを基準としている。このようなパラメータは状況によって変化すると考えられるため、動的に決定することが望ましいと考えられる。

動的に決定した方がよい例として、新な点が出現したとする移動幅が考えられる。現在、この値は7と固定値にしている。これにより、同一の点を1つの点と認識するまたは、別の点を同一の点として認識するという問題が発生している。圧力データの入力速度は使用者により変化すると考えられるため、移動幅の平均をパラメータとして使用するなどの動的な決定法が望ましいと考えられる。

\subsection{ノイズによる誤認識の解決案}
ノイズによる誤認識の解決案について述べる。現在、点情報の統合を行う際に前の圧力データの行座標のうち最も近くのものとの差を利用している。この値を点の移動幅と見なすことで、点情報が新しく出現したものか、既にある点情報が移動したものなのかを判断している。このとき、ノイズ が点と認識されているときの移動幅を図\ref{f:solution_noise}に示す。
\begin{figure}[H]
	\centering
	\begin{minipage}[b]{0.45\linewidth}
		\centering
		\includegraphics[width=6cm]{img/20180510/CSP1353000021.csv.png}
	\end{minipage}
	\begin{minipage}[b]{0.45\linewidth}
		\centering
		\includegraphics[width=6cm]{img/20180510/CSP1353000050.csv.png}
	\end{minipage}
	\caption{ノイズが認識されているときの様子}
	\label{f:solution_noise}
\end{figure}
図\ref{f:solution_noise}のオレンジ色の点がノイズを点と認識している場所である。この図より、ノイズは表示される場所は常に同じ場所であるため、移動幅が0となっていることがわかる。例外として、ノイズの出現時のみ0以外の値を取る。しかし、点情報の統合では、一定回数以上出現していない点を除去する処理があるため、移動幅が0のものは出現回数に含まないことでノイズを除去することができると考えられる。
