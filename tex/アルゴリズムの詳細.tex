本章では先行研究のアルゴリズムについて述べる。

\subsection{全体の流れ}
本節ではアルゴリズムの全体の流れについて説明する。各項目の詳細は後の節にて述べる。
先行研究のアルゴリズムの流れは
\begin{enumerate}
	\item 行座標を取得。
	\item 列座標を取得。
	\item 点情報を取得。
	\item 点情報を統合。
	\item 空行を挿入。
\end{enumerate}
である。

\subsection{使用したデータ}
点字の例として、「イチョーガヨワイ」と書かれた点字の圧力データを利用する。この点字は図\ref{f:ityo-gayowai}のような点字である。
\begin{figure}[H]
	\centering
	\includegraphics[width=16.0cm]{img/イチョーガヨワイ.png}
	\caption{「イチョーガヨワイ」の点字}
	\label{f:ityo-gayowai}
\end{figure}

\subsection{行座標の取得}
行座標の取得方法について述べる。行座標を取得する流れは
\begin{enumerate}
	\item 全ての圧力データの行方向の値を加算する。
	\item 加算した結果を式(\ref{siki:行補正})で補正する。
	\item 補正した波形から行座標を取得する。
\end{enumerate}
である。詳細について述べる。

「イチョーガヨワイ」と書かれた点字を読み込んだ際の行方向の合計値のグラフを図\ref{f:行方向の合計}に示す。
\begin{figure}[H]
	\centering
	\includegraphics[width=12.0cm]{img/row.png}
	\caption{行方向の値をすべて加算した結果}
	\label{f:行方向の合計}
\end{figure}
図\ref{f:行方向の合計}を式(\ref{siki:行補正})に従って補正する。

\begin{eqnarray}
	w &=& 0.4\sqrt{1-\left(\frac{13m}{45}\right)^2}\nonumber\\
	c_i &=& w_0b_i + \sum^3_{m=1}w_m(b_{i+m}+b_{i-m})\label{siki:行補正}
\end{eqnarray}
このとき、$i$は対象の行数を表す。補正した結果を図\ref{f:行方向の合計_補正}に示す。
\begin{figure}[H]
	\centering
	\includegraphics[width=12.0cm]{img/row_correct.png}
	\caption{行方向の合計値を補正した結果}
	\label{f:行方向の合計_補正}
\end{figure}
図\ref{f:行方向の合計_補正}の各行に対して、上下の列と比較し、いくつの列より大きい値なのかを比較する。得られた値の上位3つの値を行座標とする。この場合、47行目、32行目、17行目が行座標であることがわかる。

\subsection{列座標の取得}
列座標の取得方法について述べる。列座標を取得する流れは
\begin{enumerate}
	\item 圧力データの列方向の値の合計値を求める。
	\item 合計値を式(\ref{siki:行補正})で補正する。
	\item 補正した波形から列座標を取得する。
\end{enumerate}
である。列方向の合計値のグラフを図\ref{f:列方向の合計}に示す。
\begin{figure}[H]
	\centering
	\includegraphics[width=12.0cm]{img/col.png}
	\caption{列方向の合計値のグラフ}
	\label{f:列方向の合計}
\end{figure}
図\ref{f:列方向の合計}を式(\ref{siki:行補正})に従って補正する。このとき$i$は列を意味する。補正した結果を図\ref{f:列方向の合計_補正}に示す。
\begin{figure}[H]
	\centering
	\includegraphics[width=12.0cm]{img/col_correct.png}
	\caption{列方向の合計値を補正した結果}
	\label{f:列方向の合計_補正}
\end{figure}
図\ref{f:列方向の合計_補正}の各列に対して、左右の列と比較し、いくつの列より大きい値なのかを比較する。得られた値が3以上であれば、その列を列座標とする。

\subsection{点座標の取得}
点座標は、行座標と列座標の交点の周囲の値に重みを付けて計算する。重みを図\ref{f:重み}に示す。この値が閾値を越えればそこに点があるとする。
\begin{figure}[H]
	\centering
	\includegraphics[width=8.0cm]{img/重みのつけかた.png}
	\caption{点を認識するための重み}
	\label{f:重み}
\end{figure}

\subsection{点情報の統合}
点情報の統合では、各圧力データで求めた点座標を比較し、全体の点字を求める。点情報の統合の流れは
\begin{enumerate}
	\item すべての圧力テータに対して以下の処理を行う。
		\begin{enumerate}
			\item 前の圧力データの列座標と比較し、最も近い列座標との差を求める。
			\item これまでの差の合計値を対象の列に加算する。
		\end{enumerate}
	\item 1つ飛ばしに以下の処理を行う。
		\begin{enumerate}
			\item 前の圧力データの列座標と比較し、最も近い列座標との差を求める。
			\item 差が一定の値以上であれば新たな点、そうでなければ既に存在する点が移動したものとみなす。
			\item 現在の圧力データを除外する。
			\item これを圧力データの数が1になるまで繰り返す。
		\end{enumerate}
\end{enumerate}
である。以上の処理により、点座標を1つに統合している。

\subsection{空行の挿入}
点座標の列間距離を利用して、全体の点字を求める。列間距離の関係は図\ref{f:列間距離}のようになっている。求めた点座標の列間距離からラベルを割り合て、その関係から空行の位置を推論し、挿入する。
\begin{figure}[H]
	\centering
	\includegraphics[width=8.0cm]{img/diffs.png}
	\caption{列間距離の関係}
	\label{f:列間距離}
\end{figure}
